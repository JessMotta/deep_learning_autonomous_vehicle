\documentclass[12pt,a4paper]{article}
\usepackage[T1]{fontenc}
\usepackage[a4paper]{geometry}
\usepackage[portuges,brazilian]{babel}
\usepackage[utf8]{inputenc}
\usepackage{setspace}
\usepackage{libertine}
\usepackage{graphicx}	
\usepackage{ragged2e} 		

\begin{document} 
\begin{figure}
    \flushright
    \includegraphics[scale=0.5]{Logo_senai.png}
\end{figure}

\title{Roteiro da apresentação sobre \emph{deep learning} aplicada a veículos autônomos}
\author{Jéssica Motta\thanks{jessicalimamotta@gmail.com. SENAI-CIMATEC. CCRoSA- Centro de Competência em Robótica e Sistemas Autônomos.}}
 

    \maketitle
    \pagenumbering{arabic}
    \singlespacing

    
    \textbf{AUDIÊNCIA}

    \par As pessoas que irão assistir a apresentação são em sua maioria pessoas do meio acadêmico, professores, estudantes, pesquisadores e também pessoas que tenham interesse nesse assunto, que iniciaram o contato com a área de tecnologia ou possuem um profundo conhecimento da área com possibilidade de publicações neste âmbito. Pessoas que queiram aprofundar ou conhecer como são empregadas as técnicas de \emph{deep learning} em veículos autônomos, abrindo espaço para perguntas e contribuições por parte da audiência.


    \textbf{CONTEXTO}

    \par O tema está inserido na área de tecnologia mais especificamente no que tange a robótica, sistemas autônomos, aprendizagem de máquina e inteligência artificial. Há muito tempo vem sendo estudado o emprego de aprendizagem de máquina principalmente deep learning em veículos autônomos e isto traz questionamentos tanto da parte tecnológica quanto da parte ética.
    \par Muitos artigos e matérias que tratam desse tema abordam que a utilização desta técnica resolveria problemas no trânsito, pelo sistema apresentar uma resposta mais rápida aos estímulos e também por proporcionar autonomia para pessoas com deficiência física.
    \par Porém há muito o que pensar e estudar no que compete a este tipo de aplicação, inserir em um ambiente dinâmico, trânsito de grandes cidades, um veículo que possa identificar pessoas, animais e outros veículos e vincular suas ações ao que foi identificado requer uso de sensores e um sistema treinado, pois os elementos que aparecerão no campo de visão do veículo ocorrerão com infinitas combinações.
    \par Para se desenvolver um veículo autônomo é necessário além de materiais como câmeras, sensores e computadores embutidos para o processamento dos dados, que atendam as especificações, como também profissionais especialistas nestas áreas com habilidades para aprimorar o sistema que requer compreensão e dinamismo e não apenas a execução de tarefas pré-programadas.
    \par Guiar um veículo com segurança é algo que os seres humanos precisam de tempo de instrução, prática e adaptação. Dirigir exige conhecimentos prévios como: leis de trânsito, reconhecimento de objetos, noções de velocidade e distância, tomadas de decisões e definições de trajetos \cite{1}.
    \par A apresentação tem o intuito de tratar sobre \emph{deep learning}, como funciona a aplicação desta técnica em veículos autônomos e suscitar a discussão e participação da audiência sobre a permissão desses veículos para trafegarem no nosso dia a dia.

    \textbf{TEMPO}
    \par Haverá 20 min destinados para apresentação do conteúdo, onde os slides servirão de suporte para o assunto exposto, dedicando entre 1 min e 1:30 min para cada slide.  
    
    \textbf{PERFIL}
    \par A intenção é de passar um perfil humilde que possui certo conhecimento sobre o assunto exposto, com algumas piadas e reflexões compatíveis com o perfil do apresentador, falando pausadamente e no tom que as pessoas consigam acompanhar, e que ainda assim tem também interesse em aprender com o público. Para fazer isso a apresentação será estudada nos mínimos detalhes e ensaiada, refletir sobre o que será falado em cada slide e as possibilidades de contestação e pesquisa para responder o máximo de perguntas que surgirem.


     
    \begin{thebibliography}{BIBLIOGRAFIA}
 
        \bibitem{1} DELAI, Riccardo L.; COELHO, Alessandra Dutra. \textbf{DESENVOLVIMENTO DE VEÍCULO AUTÔNOMO-INTELIGÊNCIA CENTRAL E ORIENTAÇÃO POR CÂMERAS.}
    
    \end{thebibliography}


\end{document}
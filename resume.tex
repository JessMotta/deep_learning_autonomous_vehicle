\documentclass[12pt,a4paper]{article}
\usepackage[T1]{fontenc}
\usepackage[a4paper]{geometry}
\usepackage[portuges,brazilian]{babel}
\usepackage[utf8]{inputenc}
\usepackage{setspace}
\usepackage{libertine}
\usepackage{graphicx}	
\usepackage{ragged2e} 		

\begin{document} 
\begin{figure}
    \flushright
    \includegraphics[scale=0.5]{Logo_senai.png}
\end{figure}

\title{Roteiro da Apresentação sobre Deep learning aplicado a veículos autônomos}
\author{Jéssica Motta\thanks{jessicalimamotta@gmail.com. SENAI-CIMATEC. CCRoSA- Centro de Competência em Robótica e Sitemas Autônomos.}}
 
 

    \maketitle
    \pagenumbering{arabic}
    \singlespacing


    \textbf{AUDIÊNCIA}

    \par As pessoas que irão assistir a apresentação são em sua maioria pessoas do meio acadêmico, professores, estudantes, pesquisadores e também pessoas que tenham interesse nesse assunto, que iniciaram o contato com a área de tecnologia ou possuem um profundo conhecimento da área com possibilidade de publicações neste âmbito. Que possuam interesse em saber como são empregadas as técnicas de deep learning em veículos autônomos, abrindo espaço para perguntas e contribuições por parte da audiência.


    \textbf{CONTEXTO}
    \par O tema está inserido na área de tecnologia mais especificamente no que tange a robótica, sistemas autônomos, aprendizagem de máquina e inteligência artificial. Há muito tempo vem sendo estudado o emprego de aprendizagem de máquina principalmente deep learning em veículos autônomos e isto traz questionamentos tanto da parte tecnológica quanto da parte ética.
    \par Muitos artigos e sites que trazem esse tema abordam que a utilização desta técnica resolveria problemas no trânsito  




\end{document}
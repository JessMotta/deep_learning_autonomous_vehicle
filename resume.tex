\documentclass[12pt,a4paper]{article}
\usepackage[T1]{fontenc}
\usepackage[a4paper]{geometry}
\usepackage[portuges,brazilian]{babel}
\usepackage[utf8]{inputenc}
\usepackage{setspace}
\usepackage{libertine}
\usepackage{graphicx}	
\usepackage{ragged2e} 		

\begin{document} 
\begin{figure}
    \flushright
    \includegraphics[scale=0.5]{Logo_senai.png}
\end{figure}

\title{Roteiro da Apresentação sobre Deep learning aplicada a veículos autônomos}
\author{Jéssica Motta\thanks{jessicalimamotta@gmail.com. SENAI-CIMATEC. CCRoSA- Centro de Competência em Robótica e Sitemas Autônomos.}}
 
 

    \maketitle
    \pagenumbering{arabic}
    \singlespacing


    \textbf{AUDIÊNCIA}

    \par As pessoas que irão assistir a apresentação são em sua maioria pessoas do meio acadêmico, professores, estudantes, pesquisadores e também pessoas que tenham interesse nesse assunto, que iniciaram o contato com a área de tecnologia ou possuem um profundo conhecimento da área com possibilidade de publicações neste âmbito. Pessoas que queiram aprofundar ou conhecer como são empregadas as técnicas de deep learning em veículos autônomos, abrindo espaço para perguntas e contribuições por parte da audiência.


    \textbf{CONTEXTO}
    \par O tema está inserido na área de tecnologia mais especificamente no que tange a robótica, sistemas autônomos, aprendizagem de máquina e inteligência artificial. Há muito tempo vem sendo estudado o emprego de aprendizagem de máquina principalmente deep learning em veículos autônomos e isto traz questionamentos tanto da parte tecnológica quanto da parte ética.
    \par Muitos artigos e sites que trazem esse tema abordam que a utilização desta técnica resolveria problemas no trânsito, pelo sistema ter uma resposta mais rápida aos estímulos e também por proporcionar autonomia para pessoas com deficiência física.
    \par Porém a muito o que pensar e estudar no que compete a este tipo de aplicação, inserir em um ambiente dinâmico, trânsito de grandes cidades, um veículo que possa identificar pessoas, animais e outros veículos e vincular suas ações ao que foi identificado requer uso de sensores, uma programação com grau de falhas ínfimo e treinamento do sistema, pois os elementos que aparecerão no seu campo de visão ocorrerão com infinitas combinações.




\end{document}
\documentclass[12pt,a4paper]{article}
\usepackage[T1]{fontenc}
\usepackage[a4paper]{geometry}
\usepackage[portuges,brazilian]{babel}
\usepackage[utf8]{inputenc}
\usepackage{setspace}
\usepackage{libertine}
\usepackage{graphicx}	
\usepackage{ragged2e} 
\usepackage{adjustbox}
\usepackage{float}


\begin{document} 
\begin{figure}
    \flushright
    \includegraphics[scale=0.5]{Logo_senai.png}
\end{figure}

\title{Chamada para apresentação sobre \emph{deep learning} aplicada a veículos autônomos}
\author{Jéssica Motta\thanks{jessicalimamotta@gmail.com. SENAI-CIMATEC. CCRoSA- Centro de Competência em Robótica e Sistemas Autônomos.}}
 

    \maketitle
    \pagenumbering{arabic}
    \singlespacing


Imagine poder viajar longas distâncias ou enfrentar engarramentos estando sozinho em um carro e sem precisar dirigir. Consegue imaginar? Pois é isso que pesquisadores estão fazendo agora, procurando meios para que os carros sejam tão inteligentes, que possam dirigir sozinhos para que você possa dormir, ler, assistir um filme ou até conversar ao celular enquanto vai para o trabalho ou viaja.

Então venha assistir à palestra "A Influência da Inteligência Artificial em como nos Movemos", com a palestrante M.Engª. Jéssica Motta, do Centro de Competência em Robótica e Sistemas Autônomos- CCRoSA, SENAI- CIMATEC. Nela será elucidado sobre como a inteligência artificial, mais especificamente, a Deep Learning funciona, um breve histórico dessa tecnologia, como ela está sendo aplicada à veículos autônomos e quais as vantagens de se utilizá-la.



\end{document}
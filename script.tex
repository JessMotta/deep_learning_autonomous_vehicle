\documentclass[12pt,a4paper]{article}
\usepackage[T1]{fontenc}
\usepackage[a4paper]{geometry}
\usepackage[portuges,brazilian]{babel}
\usepackage[utf8]{inputenc}
\usepackage{setspace}
\usepackage{libertine}
\usepackage{graphicx}	
\usepackage{ragged2e} 		

\begin{document} 
\begin{figure}
    \flushright
    \includegraphics[scale=0.5]{Logo_senai.png}
\end{figure}

\title{Roteiro da apresentação sobre \emph{deep learning} aplicada a veículos autônomos}
\author{Jéssica Motta\thanks{jessicalimamotta@gmail.com. SENAI-CIMATEC. CCRoSA- Centro de Competência em Robótica e Sistemas Autônomos.}}
 

    \maketitle
    \pagenumbering{arabic}
    \singlespacing

    \section{Audiência}
    % \textbf{AUDIÊNCIA}

    \par As pessoas que irão assistir a apresentação são em sua maioria pessoas do meio acadêmico, professores, estudantes, pesquisadores e também pessoas que tenham interesse nesse assunto, que iniciaram o contato com a área de tecnologia ou possuem um profundo conhecimento da área com possibilidade de publicações neste âmbito. Pessoas que queiram aprofundar ou conhecer como são empregadas as técnicas de \emph{deep learning} em veículos autônomos, abrindo espaço para perguntas e contribuições por parte da audiência.

    \section{Contexto}
    \par A motivação de abordar este assunto encontra-se nos benefícios que os veículos autônomos podem proporcionar dando autonomia para pessoas com deficiência física e reduzindo o número de acidentes causados no trânsito pois boa parte deles são causados por sono, desatenção ou estresse. Além de tratar de um tema que está em evidência já que a Tesla prometeu ter um carro autônomo sem pedais e sem volante até 2022 e as pessoas têm a expectativa de poder se deslocar sem precisar ficar atento ao que está ocorrendo no trânsito.
  

    \section{Seções}

    \subsection{Estrutura de um veículo autônomo}
    \par Explicar de forma resumida como é composta a estrutura de um veículo autônomo, trazendo imagens para ilustrar o que está sendo dito, nivelando que já conhece do assunto e quem não conhece. 

    \subsection{Deep learning}
    \par Explicar o que é \emph{deep learning}, usando imagens e animações, sendo objetiva. Diferenciar de machine learning. E como a \emph{deep learning} está inserida nos veículos autônomos e quais os benefícios ela traz para esse tipo de sistemas. Tratar de reconhecimento de objetos e métodos de identificação. O foco maior será neste tópico pois é o ponto-chave da apresentação.

    \subsection{Citar empresas que estão construindo veículos autônomos}
    \par Fazer um comparativo do que já vem sendo feito nos últimos anos pelas empresas, tais como Tesla, Uber, Google etc. Quais os avanços que conseguiram e quais as perspectivas futuras destas empresas.

    \subsection{Levantar prós do uso dos veículos autônomos}
    \par Pontuar que a utilização destes veículos resolveria problemas no trânsito, pelo sistema apresentar uma resposta mais rápida aos estímulos e também por proporcionar autonomia para pessoas com deficiência física.
    \par Guiar um veículo com segurança é algo que os seres humanos precisam de tempo de instrução, prática e adaptação. Dirigir exige conhecimentos prévios como: leis de trânsito, reconhecimento de objetos, noções de velocidade e distância, tomadas de decisões e definições de trajetos \cite{1}. 
    \par Abordar que o emprego da deep learning em veículos autônomos reduziria a quantidade de acidentes já que esse tipo de \emph{machine learning} é focado em identificar objetos e realizar previsões.

    \subsection{Contar uma história}
    \par Neste momento fazer o link com o público através da identificação. Trazer a história que assim como eu muitas pessoas trabalham em cidades diferentes das quais residem e que muitas vezes fazem viagem de horas para realizarem esse trajeto, cansadas e com outras demandas como saber como a família está e resolver problemas. Trazer dados sobre o número de acidentes anuais no trânsito \textit{versus} a quantidade de acidentes causados pelos carros autônomos.


    \section{Tempo}
    \par Haverá 20 min destinados para apresentação do conteúdo, onde os slides servirão de suporte para o assunto exposto, dedicando entre 1 min e 1:30 min para cada slide.  
    
    \section{Perfil}
    \par A intenção é de passar um perfil que possui certo conhecimento sobre o assunto exposto, falando pausadamente e no tom que as pessoas consigam acompanhar, e que ainda assim tem também interesse em aprender com o público. Para fazer isso a apresentação será estudada nos mínimos detalhes e ensaiada, refletir sobre o que será falado em cada slide e as possibilidades de contestação e pesquisa para responder o máximo de perguntas que surgirem.


     
    \begin{thebibliography}{BIBLIOGRAFIA}
 
        \bibitem{1} DELAI, Riccardo L.; COELHO, Alessandra Dutra. \textbf{DESENVOLVIMENTO DE VEÍCULO AUTÔNOMO-INTELIGÊNCIA CENTRAL E ORIENTAÇÃO POR CÂMERAS.}
    
    \end{thebibliography}

    % https://medium.com/brasil-ai/como-funcionam-os-carros-aut%C3%B4nomos-parte-1-sensoriamento-e-vis%C3%A3o-computacional-ae25d17c66a1

    % https://tecnoblog.net/286916/tesla-carros-autonomos-robotaxis-fsd/

\end{document}